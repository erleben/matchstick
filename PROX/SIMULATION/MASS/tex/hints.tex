\documentclass[twocolumn]{article}

\usepackage{graphicx}
\usepackage{subfigure}
\usepackage{amsmath}
\usepackage{amsfonts}
\usepackage[usenames,dvipsnames]{color} % good support for colors
\usepackage{pstricks}

\def\registered{\textsuperscript{\textregistered}}
\def\copyright{\textsuperscript{\textcopyright}}
\def\trademark{\textsuperscript{\texttrademark}}

\newrgbcolor{kennycolor}{.1 .1 .7}
\newcommand{\kenny}[1]{  { \bf \kennycolor Kenny:  #1}}

\renewcommand{\vec}[1]{ \ensuremath{\mathbf{#1} } }
\newcommand{\abs}[1]{\left| {#1} \right|}
\newcommand{\quat}[1]{ #1 }
\newcommand{\mat}[1]{\ensuremath{\mathbf{#1} }}
\newcommand{\norm}[1]{\parallel {#1} \parallel}
\renewcommand{\Re}{ \mathbb{R} }
\newcommand{\set}[1]{\mathcal{#1}}
\newcommand{\identity}{ \ensuremath{ \mat I }}
\newcommand{\bigO}{ \ensuremath{ \mathcal{O} }}
\newcommand{\atan }{ \ensuremath{ \phantom{\cdot}\text{atan}_2 } }
\newcommand{\acos }{ \ensuremath{ \cos^{-1} } }
\newcommand{\asin }{ \ensuremath{ \sin^{-1} } }
\newcommand{\logand }{ \ensuremath{ \wedge } }
\newcommand{\logor }{ \ensuremath{ \vee } }
\newcommand{\degree}{\,^{\circ}}
\renewcommand{\th}{ \ensuremath{ ^{\text{th}}} }
\newcommand{\sgn }[1]{ \ensuremath{ \text{sgn}\left( #1 \right) } }
\newcommand{\union}{\ensuremath{ \cup }}
\newcommand{\intersection}{\ensuremath{ \cap }}
\newcommand{\hull }[1]{ \ensuremath{ \text{convex hull}\left( #1 \right) } }
\newcommand{\proj }[1]{ \ensuremath{ \text{proj}\left( #1 \right) } }
\newcommand{\trace }[1]{ \ensuremath{ \text{tr}\left( #1 \right) } }
\newcommand{\inertia}{\ensuremath{\mathcal{I} }}
\newcommand{\mass}{\ensuremath{\mathcal{M} }}

\title{Hints for using the MASS Library}
\author{Kenny Erleben}
\date{December 2010}

\begin{document}

\maketitle

\begin{abstract}
  The MASS library was created to provide a functional toolbox for handling mass
  properties of rigid bodies. This includes functions for computing the total
  mass of an object $\mass$ the center of mass position $\vec r$ and the inertia
  tensor $\inertia$. These values depend on the chosen frame of reference. To
  achieve a correct physical simulation it is important to apply the correct
  mass values. In this short paper we will address the issues in computing the
  values of mass properties of rigid bodies and compounds hereof. Our
  contribution includes a description of how to deal with initialization and the
  visualization synchronizations of rigid bodies and the construction of
  compound bodies.
\end{abstract}

\section*{Creation of a Rigid Body}
When creating content in rigid body simulators several different coordinate
frames are used. We will work with three different coordinate frames
\begin{description}
\item[The world frame] is the world coordinate frame wherein everything can be
  absolutely placed.
\item[The body frame ] refers to the coordinate frame with its origin at the
  center of mass of a given rigid body and the orientation of the frame chosen
  such that the inertia tensor of the given body is a constant and diagonal
  tensor.
\item[The model frame] is some convenient frame used to describe the geometry
  in. The frame is fixed in the sense that it always will follow the same motion
  as the geometry.
\end{description}
The model frame needs not be physically founded, but can in principle be defined
anywhere where it make sense from an artistic or modelling viewpoint. From a
modelling viewpoint one would define the geometry in some model reference frame
and then place the model frames in the world frame. However, when connecting to
a rigid body simulator one must account find the body frame as this is the one
the simualtor works with. This leads to the following steps of actions
\begin{enumerate}
\item Compute model frame values $\mass$, $\vec r_M$, and $\inertia_M$
\item Transform from model frame values into the body frame values $\inertia_B$.
\item Given model frame placement in the world $\vec r$ and $\mat R$ compute the
  body frame placement in the world $\vec r_B$ and $\mat R_B$.
\end{enumerate}
The MASS library provides routines for the first step. For the second step one
would have to first transform the reference point to the center of mass position
using the parallel-axis theorem ($\textbf{p-a-t}$)\footnote{See
  $mass::translate_inertia$ function},
\begin{equation}
  \inertia^\prime = \textbf{p-a-t}(\inertia_M, - \vec   r_M) .
\end{equation}
Next we can apply eigen-value-decomposition to find the body frame
values\footnote{See $mass::compute_orientation$ function},
\begin{equation}
  \inertia_M  = \mat R_M \inertia_B \mat R_M^T.
\end{equation}
Now given the model frame placement in the world frame we compute,
\begin{subequations}
  \begin{align}
    \vec r_B &= \vec r - \mat R \vec r_M, \\
    \mat R_B &= \mat R \mat R_M.
  \end{align}
\end{subequations}
There is one more subtlety in the creation process. The rigid body simulator
must know where its geometry is relative to its body frame. Thus, one should
apply the transformation from model frame to body frame onto the geometry. That
is first translate by $- \vec r_M$ and then rotation by $\mat
R_M^{-1}$\footnote{Some simultors allow one to specify these transformations
  directly. Other simulators assume geometry is living in body space in which
  case one actually has to transform the geometry.}.

\section*{Visualization Update of a Rigid Body}\label{sec:visualization-update}
During simulation a rigid body simulator will compute new values of $\vec r_B$
and $\mat R_B$. However, for the visualization one needs to find the new
placement of the model frame in the world. Thus one must solve
\begin{subequations}
  \begin{align}
    \vec r  &= \vec r_B + \mat R \vec r_M, \\
    \mat R &= \mat R_B \mat R_M^{-1}.
  \end{align}
\end{subequations}
Thus, one must always store the body frame to model frame transformation given
by $\vec r_M$ and $\mat R_M$.


\section*{The Curious Parallel Axis Theorem}\label{sec:curi-parall-axis}
One has to be carefull when applying the parallel axis theorem. By its
definition it transforms a body frame inertia tensors $\inertia$ into a modified
tensor $\inertia^\prime$
having the same orientation but a different reference point given by the
translation $\vec d$
\begin{subequations}
  \begin{align}
    \inertia_{\alpha \alpha}^\prime 
    &=
    \inertia_{\alpha \alpha} + \mass (\vec d_{\beta}^2  + \vec d_{\gamma}^2)\\
    \inertia_{\alpha \beta}^\prime &=  \inertia_{\alpha \beta} - \mass (\vec
    d_{\alpha} \vec d_{\beta})
  \end{align}
\end{subequations}
Here the primed quantities would be the modified inertia tensor that no longer
lives in the body space frame. The translation given here is the vector from the
body space frame origin to the new origin of the new model frame.

However, here comes the tricky part say one which to do another translation to
get $\inertia^{\prime\prime}$ then one can not just apply the above formula to
$\inertia^\prime$. Instead one must first transform $\inertia^\prime$ back to the
body space frame using 
\begin{subequations}
  \begin{align}
    \inertia_{\alpha \alpha} 
    &=
    \inertia_{\alpha \alpha}^\prime - \mass (\vec d_{\beta}^2  + \vec d_{\gamma}^2)
    \\
    \inertia_{\alpha \beta} 
    &=
    \inertia_{\alpha \beta}^\prime + \mass (\vec d_{\alpha} \vec d_{\beta})
  \end{align}
\end{subequations}
and first then may one apply the transform taking one from the body space
inertia tensor into the $\inertia^{\prime\prime}$ tensor.

\section*{Creating Compounds}\label{sec:creating-compounds}
As mass properties are defined by volume integrals it is easy to see that all we
need to do is to make sure all properties are given with respect to the same
reference frame. When this is the case we can simply sum up all the mass
properties.

Here we will give an example of two rigid bodies their body frame inertia
tensors are given by the constant diagonal tensors $\inertia_A$ and
$\inertia_B$. The center of mass positions are given by $\vec r_A$ and $\vec
r_B$ and the orientatio of the body frames wrt. the world frame is given by
$\mat R_A$ and $\mat R_B$.

In our first step we will transform inertia tensors into the world frame
\begin{subequations}
\begin{align}
  \inertia^{\prime}_A &\leftarrow \mat R_A \inertia_A \mat R_A^T,\\
  \inertia^{\prime}_B &\leftarrow \mat R_B \inertia_B \mat R_B^T,\\
  \inertia^{\prime^\prime}_A &\leftarrow \textbf{p-a-t}(\inertia_A^\prime,  \vec   r_A) ,\\
  \inertia^{\prime^\prime}_B &\leftarrow \textbf{p-a-t}(\inertia_B^\prime, \vec
  r_B) .
\end{align}  
\end{subequations}
Next we may find the compund inertia tensor with reference to the world frame
\begin{equation}
  \inertia_C^\prime = \inertia^{\prime^\prime}_A + \inertia^{\prime^\prime}_B
\end{equation}
The total mass is simply
\begin{equation}
  \mass_C = \mass_A + \mass_B
\end{equation}
and the center of mass position is
\begin{equation}
  \vec r_C =  \frac{\mass_A \vec r_A + \mass_B \vec r_B}{ \mass_C}
\end{equation}
What remains is to find the body frame inertia tensor
\begin{subequations}
  \begin{align}
    \inertia_C^{\prime\prime} &\leftarrow \textbf{p-a-t}(\inertia_C^\prime, - \vec   r_C) .\\
    \mat R_C \inertia_C \mat R_C^T  &\leftarrow \inertia_C^{\prime\prime}
  \end{align}
\end{subequations}
The recipe can be incremental extended straigthforwardly. Observe that there is
one more snag as with the creation of rigid bodies one must ensure that
geometries in the rigid body simulator is given writh respect to the new
compound bodies body frame.




\section*{Handling a Deformed Box}\label{sec:handl-deform-box}
Imagine we are given a deformed box shape. We assume that the deformation can be
specified by some linear coordinate transformation and write thie mathematically
as
\begin{equation*}
  \begin{bmatrix}
    x\\
    y\\
    z
  \end{bmatrix}
  =
  \mat \Phi
  \begin{bmatrix}
    X\\
    Y\\
    Z
  \end{bmatrix}
\end{equation*}
where the linear transformation is given by the matrix $\mat A$ that maps
coordinates the $(X, Y, Z)$ from a regular unit box with its center placed at
the origin into the deformed coordinates $(x, y, z)$. Here we assume that the
mapping is bijective and thus $\mat  \Phi$ must be invertible implying we
can find the inverse mapping given by $\mat \Phi^{-1}$

We now wish to find closed form solutions for the mass properties of the
deformed box. That is we wish to solve the volume integrals
\begin{subequations}
  \begin{align*}
    \mass &= \int_v  \rho   dv \\
    \vec r_x &= \frac{1}{\mass}\int_v  \rho \vec x  dv \\
    \vec r_y &= \frac{1}{\mass}\int_v  \rho \vec y  dv \\
    \vec r_z &= \frac{1}{\mass}\int_v  \rho \vec z  dv \\
    \inertia_{xx} &= \int_v  \rho \left( y^2 + z^2 \right)  dv \\
    \inertia_{yy} &= \int_v  \rho \left( x^2 + y^2 \right)  dv \\
    \inertia_{zz} &= \int_v  \rho \left( x^2 + y^2 \right)  dv \\
    \inertia_{xy} &= - \int_v  \rho \left( x y \right)  dv \\
    \inertia_{xz} &= - \int_v  \rho \left( x z\right)  dv \\
    \inertia_{yz} &= - \int_v  \rho \left( y z \right)  dv
  \end{align*}
\end{subequations}
Our approach to finding the closed form solutions we seek is to rewrite the
volume integrals such that we are integrating over the undeformed volume. The
machinery is pretty much the same for all equations so we will here just treat
one term in detail and leave the remaining terms for the reader.

First we will make a change of variables using the formula $dv = j dV$ where $j
= \det(\mat  \Phi)$
\begin{subequations}
  \begin{align*}
    \inertia_{xx} &= \int_v  \rho \left( y^2 + z^2 \right)  dv \\
    \inertia_{xx} &= j \rho \int_V   \left( y^2 + z^2 \right)  dV
  \end{align*}  
\end{subequations}
Secondly we observe that the deformed coordinates are a linear mapping of
undeformed coordinates that means
\begin{subequations}
  \begin{align*}
    x  &=  \Phi_{11} X + \Phi_{12} Y + \Phi_{13} Z\\
    y  &=  \Phi_{21} X + \Phi_{22} Y + \Phi_{23} Z\\
    z  &=  \Phi_{31} X + \Phi_{32} Y + \Phi_{33} Z
  \end{align*}
\end{subequations}
using this we have that
\begin{equation*}
  \begin{split}
    y^2 + z^2 
    =
    \left(\Phi_{21} X + \Phi_{22} Y + \Phi_{23} Z\right)^2 
    \quad \quad \quad \quad \quad 
    \\
    +
    \left(\Phi_{31} X + \Phi_{32} Y    + \Phi_{33} Z\right)^2 
    =
    P_2(X,Y,Z)    
  \end{split}
\end{equation*}
where $P_2$ denotes a general second order polynomial in $X$, $Y$ and $Z$. So
now the volume integral reads
\begin{subequations}
  \begin{align*}
    \inertia_{xx} 
    &=
    j \rho \int_V   P_2(X,Y,Z)  dV\\
    &= 
    j \rho 
    \int_{-\frac{1}{2}}^{\frac{1}{2}}  
    \int_{-\frac{1}{2}}^{\frac{1}{2}}  
    \int_{-\frac{1}{2}}^{\frac{1}{2}}  
    P_2(X,Y,Z)
    dX
    dY
    dZ
  \end{align*}
\end{subequations}
Which is straightforward to solve for a closed form solution. All that are
missing in order to do this is formulas for the coefficients of $P_2$ in the
given integral these are
\begin{subequations}
  \begin{align*}
    P_2(X,Y,Z) 
    &= \underbrace{ \left (\Phi_{21}^2 + \Phi_{31}^2\right) }_{a_{XX}} X^2 \\
    &+ \underbrace{ \left (\Phi_{22}^2 + \Phi_{32}^2\right) }_{a_{YY}}  Y^2 \\
    &+ \underbrace{ \left (\Phi_{23}^2 + \Phi_{33}^2\right) }_{a_{ZZ}}  Z^2\\
    &+ \underbrace{ 2 \left ( \Phi_{21} \Phi_{22} + \Phi_{31} \Phi_{32} \right)
    }_{a_{XY}}  X Y \\
    &+ \underbrace{ 2 \left ( \Phi_{21} \Phi_{23} + \Phi_{31} \Phi_{33} \right)
    }_{a_{XZ}}  X Z \\
    &+ \underbrace{ 2 \left(\Phi_{22} \Phi_{23} + \Phi_{32} \Phi_{33}\right) }_{a_{YZ}} Y Z 
  \end{align*}
\end{subequations}
From this we find
\begin{subequations}
  \begin{align*}
    \inertia_{xx}
    =
    j \rho
    \begin{bmatrix}
      \begin{matrix}
        \frac{a_{XX}}{3} X^3 Y Z +\frac{a_{yy}}{3} X Y^3 Z +\\
        \quad \frac{a_{zz}}{3} X Y Z^3 +  \frac{a_{XY}}{4} X^2 Y^2  Z +\\
        \quad \quad         \frac{a_{XZ}}{4} X^2 Y  Z^2 + \frac{a_{YZ}}{4} X Y^2  Z^2      
      \end{matrix}
    \end{bmatrix}_{ (-\frac{1}{2}, -\frac{1}{2}, -\frac{1}{2} ) } ^{ (
      \frac{1}{2}, \frac{1}{2}, \frac{1}{2})  }
  \end{align*}
\end{subequations}
From this we can generalize the recipe for deriving a closed form solution for
all the volume intergrals. The other principal moments of inertia will be
permutations of the above derivation. The products of moments are all similar
and will also result in integrals of second order polynomials. The center of
mass integrals all results in integrals of linear polynomials and the mass
volume integral is straightforward.




\bibliographystyle{plain}
\bibliography{references}


\end{document}

%%% Local Variables: 
%%% mode: pdflatex
%%% TeX-master: t
%%% End: 
